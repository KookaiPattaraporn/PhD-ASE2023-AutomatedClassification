\section{Conclusion and Future Work} \label{sec:conclusion}

User privacy in software development has attracted attention from the software development community in the past recent years. Together with the enactment of data protection regulations and laws, it is essential for software development teams to properly address privacy concerns in their software systems. In this paper, we have explored a wide range of textual feature techniques that can be used to automatically classify privacy requirements in issue reports. We performed our study on issue reports of Google Chrome and Moodle. We used MRR and recall@k to evaluate the performance of these techniques. The evaluation results showed that BoW, N-gram IDF, TF-IDF, Word2Vec are suitable for privacy requirements classification. In addition, N-gram IDF is the best performer in our experiment. We believe that this study provides insightful reference in choosing textual feature extraction techniques in future work. Our future work involves \newtext{studying on the robustness performance and in-depth analysis of different textual feature extraction techniques}. We plan to extend our study on privacy-related issue reports of other software projects (e.g. health and mobile applications). We also plan to explore this line of research further and develop it as an automated tool to support the privacy requirements identification in issue tracking systems (e.g. JIRA).

%Our evaluation shows that BERT performs really well on classifying issue reports into relevant privacy requirements with all the measures (i.e. accuracy, precision, recall and F-measure) above 95\%. 