\section{Introduction} \label{sec:intro}

In this digital era, software applications play an important role in providing a range of services for people in their daily lives (e.g. browsing information and using online learning platforms). People leave their digital footprints such as search history and personal details when interacting with those software applications. This raises critical concerns to their privacy and personal data protection. There are many data protection and privacy legislations and policies around the world put in place to govern personal data processing (e.g. General Data Protection Regulations (EU) \cite{OfficeJournaloftheEuropeanUnion;2016} and California Consumer Privacy Act (US) \cite{StateofCaliforniaDepartmentofJustice2018}). These regulations provide a set of requirements for handling personal data in organisations. They also provide the rights for individuals to manage their personal data (e.g. right to be informed and right of access). The organisations needing to comply with these regulations are required to consider privacy compliance in their software systems. Failing to comply with these regulations may cause negative consequences to organisations in terms of reputation and financial hardship \cite{Data, EuropeanCommission2019, CNET, PrivacyAffa}. 

In addition, the cases of privacy breaches and vulnerabilities have been rapidly increasing. Those breaches not only occurred in small organisations but also happened to the world's top leading companies (e.g. Google and Marriott) \cite{InformationCommissionersOffice2020, ICOMarriott, Swinhoe2020}. These scenarios affect the organisations' reputation and exposed individuals. Hence, there is an urgent need to ensure that privacy and personal data protection are taken into consideration when developing software systems. It is however challenging for organisations to integrate privacy and personal data protection requirements into the existing processes, especially for deployed systems.

As an agile, issue-driven software development approach has been increasingly adopted in most today's software projects, issues become a main source of requirements of the software project \cite{Choetkiertikul}. Issues are lodged into issue tracking systems (e.g. JIRA) which are accessible for all the stakeholders who involve in the projects. For each development iteration, the development team selects a set of issues to work on for that iteration. Hence, the issue reports are the first source of requirements and project tasks that are considered by the development team. Issues contain important information about new requirements (i.e. feature requests), change requests for existing requirements (i.e. improvements), reporting requirements not being properly met (i.e. bugs) or representing work that has to be done (i.e. tasks) \cite{Choetkiertikul2018, Choetkiertikul}. 

\begin{figure}[H]
	\centering
	\includegraphics[width=1\linewidth]{"Figures/issue_ex"}
	\caption{An example of Google Chrome (top) and Moodle (bottom) issue reports.}
	\label{fig:issue-example}
\end{figure}

Issue reports are normally written in \emph{natural language} (see Figure \ref{fig:issue-example}). They contain information (e.g. issue key or ID, issue summary, issue description, issue type, priority, status and components) that describes scenarios and states actions needed to be attend by software engineers. Given thousands of issue reports in large software projects, it is challenging for software engineers to identify privacy-related issues. In addition, privacy concerns vary depending on functionalities and context provided by a software. For example, a web browser (e.g. Chrome) may prioritise the user search history while an online learning platform (e.g. Moodle) focuses on protecting user profiles and personal information. Hence, there is an emerging need to identify relevant privacy requirements in issue reports.

\newtext{A recent study developed a taxonomy of privacy requirements from data protection regulations and privacy frameworks \cite{Sangaroonsilp2023}. This taxonomy provides a set of fundamental privacy requirements for developing privacy-aware software applications. One of its usages is to classify privacy-related issues in a software project into relevant privacy requirements. This classification facilitates software development teams in identifying privacy requirements concerned in issue reports as well as ensuring that the associated privacy requirements are properly addressed. In addition, the process would also enable privacy compliance checking which requires a demonstration on the privacy needs and concerns in legislations are addressed in a software system. Since the taxonomy contains a large group of privacy requirements, the classification process is labour intensive and time consuming. Hence, an \emph{automated support} is much needed for performing this task. The support could be provided in a ``just-in-time'' manner: once an issue is created or modified, the machinery (integrated with an issue-tracking system) will automatically classify the issue into appropriate privacy categories and requirements.}

One prominent option to develop this automated solution is leveraging the usage of machine learning (ML) and natural language processing (NLP) techniques. Information in an issue report (e.g. title and description) is extracted into features. Those features are then used to build machine learners that are capable of learning from historical data to perform the classification on new data. There is a range of state-of-the-art techniques for extracting and learning those so-called textual features. In this study, we explore a wide range of machine learning and natural language processing techniques that can be used to automatically classify privacy requirements in issue reports. \newtext{This paper provides the following contributions:} \\

\begin{itemize}
	
	\item \newtext{We evaluate the performance of the traditional word embedding techniques (i.e. BoW \cite{Tirilly2008}, N-gram IDF \cite{Shirakawa2015}, TF-IDF \cite{blei2003latent} and Word2Vec \cite{Mikolov2013, Google2013}) and deep learning techniques (i.e. CNN \cite{LeeMin-JaeHeoChan-GunLee2017} and BERT \cite{Devlin2018}) in classifying privacy requirements in issue reports. We use the labelled dataset published by \citeauthor{Sangaroonsilp2023} \cite{Sangaroonsilp2023} as input data in this empirical study. We employ \emph{Mean Reciprocal Rank (MRR)} and \emph{recall at k (Recall@k)} to compare the performance of each method. In addition, we identify the best performing technique that can be used to assist a software team in identifying privacy requirements in issue reports. The results confirmed that N-gram IDF is the best performer with 0.6093 and 0.5838 on MRR in Google Chrome and Moodle projects respectively. TF-IDF also performs well on both MRR and recall@5 in both projects. It achieves 0.6093 on MRR in Google Chrome project and achieve the highest recall@5 at 0.7866 and 0.6027 in Google Chrome and Moodle respectively.}
	
	\item \newtext{We perform a Wilcoxon test to investigate if the classification results of all classification methods are statistically significant difference. We found that the recall@k results of random guessing method are statistically significant difference for all the traditional word embedding and deep learning techniques with effect sizes greater than 0.95 in Google Chrome project. In Moodle project, only the recall@k results between random guessing and BoW, N-gram IDF and TF-IDF, and between N-gram IDF and BERT are statistically significant difference with effect sizes greater 0.95.} \\
	
\end{itemize}

A full replication package containing all the artefacts generated by our studies is publicly made available at \cite{msrreplipkg}. The remainder of this paper is structured as follows. We introduce background and related work on a taxonomy for privacy requirements classification and text classification techniques in Section \ref{sec:related-work}. In Section \ref{sec:approach}, we discuss on a motivating example and explain the approaches implemented in our study. The details of dataset, experimental setting, performance measures and evaluation results are presented in Section \ref{sec:evaluation}. We address threats to validity of our study in Section \ref{sec:threats}. Finally, we conclude and discuss future work in Section \ref{sec:conclusion}.
